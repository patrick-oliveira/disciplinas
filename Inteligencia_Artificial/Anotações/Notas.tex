\documentclass{article}
\usepackage[portuguese]{babel}
\usepackage[utf8]{inputenc}
\usepackage{amsmath, amsthm, amssymb, amsfonts}
\usepackage{subcaption}
\usepackage{mathtools}
\usepackage{graphicx}
\usepackage{color}
\usepackage{authblk}
\usepackage[colorlinks,citecolor=red,urlcolor=blue,bookmarks=false,hypertexnames=true]{hyperref}
\usepackage{geometry}
\usepackage{float}
\geometry{
     a4paper,
     total={170mm,257mm},
     left=20mm,
     top=20mm,
}


\newcommand{\limite}{\displaystyle\lim}
\newtheorem{teorema}{Teorema}[section]
\newtheorem{definicao}{Definição}[section]
\newtheorem{proposicao}{Proposição}[section]
\newtheorem{corolario}[teorema]{Corolário}
\newtheorem{lema}{Lema}[section]
\newtheorem{exemplo}{Exemplo}[section]
\newtheorem{theorem}{Theorem}[section]
\newtheorem{definition}{Definition}[section]
\newtheorem{collorary}[theorem]{Collorary}
\newtheorem{lemma}{Lemma}[section]
\newtheorem{proposition}{Propositon}[section]
\newtheorem{example}{Example}[section]


% Dados de identificação
\title{Notas do Curso}
\author{Patrick Oliveira}
\affil{Curso ministrado pelo Prof. Denis Fantinato, 3Q19. As notas envolvem anotações de aula e recortes de livros e artigos.} 


\begin{document}
\maketitle


Different definitions and approaches to AI.

\begin{enumerate}
    \item \textbf{Thinking Humanly}: "[The automation of] activities that we associate with human thinking, activities such as decision-making, problem solving, learning..." (Bellman, 1978)
    \item \textbf{Thinking Rationally}: "The study of the computations that make it possible to perceive, reason, and act." (Winston, 1992)
    \item \textbf{Acting Humanly}: "The art of creating machines that perform functions that require intelligence when performed by people." (Kurzweil, 1990)
    \item \textbf{Acting Rationally}: "Computational Intelligence is the study of the design of intelligent agents." (Poole \textit{et al.}, 1998)
\end{enumerate}

\section{Intelligent Agents}

An \textit{agent} is anything that can be viewed as perceiveing its \textit{environment} through \textit{sensors} and acting upon that environment through \textit{actuators}. We use the term \textit{percept} to refer to the agent's perceptual inputs at any given instant. An agent's \textit{percept sequence} is the complete history of everything the agent has ever perceived. Mathematically speaking, we say that an agent's behavior is described by the \textit{agent function} that maps any given percept sequence to an action.

The correct action is decided based on its consequences. When an agent is plunked down in an environment, it generates a sequence of actions according to the percepts it receives. This sequence causes the environment to go through a sequence of states. If the sequence is desirable, then the agent has performed well. This notion of desirability is captured by a \textit{performance measure} that evaluates any given sequence of environment variables.

What is rational at any given time depends on four things:

\begin{itemize}
    \item The performance measure that defines the criterion of success.
    \item The agent's prior knowledge of the environment.
    \item The actions that the agent can perform.
    \item The agent's percept sequence to date.
\end{itemize}

This leads to a definition of rational agent: \textit{For each possible percept sequence, a rational agent should select an action that is expected to maximize its performance measure, given the evidence provided by the percept sequence and whatever built-in knowledge the agent has.}

\section{Solving Problems By Searching}





\end{document}